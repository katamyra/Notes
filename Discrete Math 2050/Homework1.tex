\documentclass[11pt]{article}
\usepackage[utf8]{inputenc}
\usepackage{setspace}
\setlength{\parindent}{0px}
\usepackage{sqrcaps}
\usepackage[T1]{fontenc}
\usepackage{cancel}
\usepackage{framed}
\usepackage{mdframed}
\usepackage{graphicx, tipa}
\usepackage[dvipsnames]{xcolor}
\usepackage{asymptote}
\usepackage{hyperref}
\usepackage{fancyhdr}
\usepackage{amsfonts}
\usepackage[document]{ragged2e}
\usepackage[total={6.5in, 8in}]{geometry}
\usepackage{amsmath, amsfonts, amssymb, mathtools, versions}
\usepackage{fourier}
\usepackage{adjustbox}
\usepackage{hyperref}
\usepackage{float}
\usepackage{anyfontsize}
\usepackage{mdframed}
\begin{asydef}
usepackage("fouriernc");
\end{asydef}
\newcommand{\overarc}[1]{{\setbox9=\hbox{#1}\ooalign{\resizebox{\wd9}{\height}{\
texttoptiebar{\phantom{A}}}\cr#1}}}
\DeclareMathOperator{\arcsec}{arcsec}
\DeclareMathOperator{\arccot}{arccot}
\DeclareMathOperator{\arccsc}{arccsc}
\DeclareMathOperator{\lcm}{lcm}
%For bolded items in enumerate
\newenvironment{enumbf}{\begin{enumerate}[font=\textbf]}{\end{enumerate}}
%For answer choices
\newcommand{\ans}[5]{\bigskip
$\textbf{(A)}\ #1 \qquad\textbf{(B)}\ #2 \qquad\textbf{(C)}\ #3 \qquad\textbf{(D)}\
#4 \qquad\textbf{(E)}\ #5$}
%For colored boxed answer
\newcommand{\bboxed}[1]{\color{maincolor}\boxed{#1}}
%For solutions
\newenvironment{solution}
{
\vspace{0.5mm}
\begin{mdframed}
[linewidth=3pt,leftline=true,rightline=false,bottomline=false,topline=false,linecol
or=maincolor,backgroundcolor=maincolor!10]\color{maincolor}\textbf{\
textsf{Solution: }}\color{black}
}
{\color{black}\end{mdframed}\vspace{0.5mm}}
%For alternate solutions
\newcommand{\orr}{
\begin{center}
\textbf{OR}
\end{center}
}
%For remarks
\newcommand{\bbold}[1]{\color{maincolor}\textbf{\textsf{#1} }\color{black}}
%For problem proposers
\newcommand{\prop}[1]{\color{maincolor}\textbf{\textsf{(#1)} }\color{black} }
%For headings (manual bigskips)
\newcommand{\bhead}[1]{\textbf{\textsf{\Large #1}}\normalsize}
%For headings (automatic bigskips)
\newcommand{\bbhead}[1]{\bigskip\textbf{\textsf{\Large #1}}\normalsize\bigskip}
\newcommand{\pt}[1]{\textcolor{maincolor}{(#1 points)}}
\newcommand{\pte}[1]{\textcolor{maincolor}{(#1 points each)}}
\title{CS 2050 Fall 2023 Homework 1}
\author{Due: September 1 @ 11:59 PM}
\date{Released: August 25}
\definecolor{maincolor}{RGB}{114, 140, 255}
\begin{document}
\maketitle
\begin{justify}
This assignment is due at \textbf{11:59 PM EDT} on \textbf{Friday, September 1,
2023}. Submissions submitted at least 24 hours prior to the due date will receive
2.5 points of extra credit. On-time submissions receive no penalty. You may turn it
in one day late for a 10-point penalty or two days late for a 25-point penalty.
Assignments more than two days late will NOT be accepted. We will prioritize on-
time submissions when grading before an exam.
\bigskip
You should submit a typeset or \emph{neatly} written PDF on Gradescope. The
grading TA should not have to struggle to read what you've written; if your
handwriting is hard to decipher, you will be required to typeset your future
assignments. Illegible solutions will be given 0 credit. A 5-point penalty will
occur if pages are incorrectly assigned to questions when submitting.
\bigskip
You may collaborate with other students, but any written work should be your own.
Write the names of the students you work with on the top of your assignment.
\bigskip
Always justify your work, even if the problem doesn't specify it. It can help the
TA's to give you partial credit.
\bigskip
Author(s): Ronnie Howard
\clearpage
\begin{enumerate}
\item \pte{2} Rewrite each of the following in the form ``if \ldots, then \
ldots". (You may adjust verb tense as you wish to make the sentences sound
natural.)
\begin{enumerate}
\item Rowing a boat is necessary for being a Viking.
\begin{mdframed}
    If you are a Viking, you row a boat.
\end{mdframed}
\item You sharpen your axe unless you are not a Viking.
\begin{mdframed}
If you are a viking, you sharten your axe.
\end{mdframed}
\item You are a Saxon provided that you are from England.
\begin{mdframed}
If you are from England, you are a Saxon
\end{mdframed}
\end{enumerate}
\item \pte{2} Evaluate each of the following propositions as True or False.
\begin{enumerate}
\item If humans are birds, then humans are birds.
\begin{mdframed}
    True.
\end{mdframed}
\item If $2 \leq 3$, then $2 > -1$.
\begin{mdframed}
   True. 
\end{mdframed}
\item If $4^2 = 33$, then $7-5=100$.
\begin{mdframed}
    True.
\end{mdframed}
\item If $8$ is even, then $10$ is odd.
\begin{mdframed}
    False.
\end{mdframed}
\end{enumerate}
\item \pte{2} Let $s$ be the proposition ``You are a pirate," let $d$ be the
proposition ``You are on a pirate ship," and let $n$ be the proposition ``It is
storming." Express the following as English sentences. (You may adjust tense as you
like.)
\begin{enumerate}
\item $(d \lor n) \rightarrow s$
\begin{mdframed}
    If you are on a pirate ship or it is storming, then you are a pirate.
\end{mdframed}
\item $\lnot d \rightarrow (n \rightarrow s)$
\begin{mdframed}
    If you are not on a pirate ship, then if it is storming, you are a pirate.
\end{mdframed}
\item $(n \lor d) \leftrightarrow s$
\begin{mdframed}
    It is storming or you are on a pirate ship if and only if you are a pirate.
\end{mdframed}
\end{enumerate}
\item \pte{3} Let $l$ be the proposition ``You walk the plank," let $a$ be the
proposition ``You upset the captain," and let $h$ be the proposition ``You raised
the sail." Represent each of the following statements using only $l$, $a$, $h$, and
logical operators. Then, negate the statements you identify pushing all negations
in as far as possible. Then, translate it back to English.
\begin{enumerate}
\item You upset the captain only when you walk the plank.
\begin{mdframed}
\begin{align*}
    &a \rightarrow l && \text{given}\\
    &\lnot (\lnot a \lor l) && \text{negation} \\
    &a \land \lnot l && \text{De Morgans Law}\\
    &\text{You upset the captain and you did not walk the plank} \\
\end{align*}
\end{mdframed}
\item You do not walk the plank unless you did not raise the sail.
\begin{mdframed}
\begin{align*}
    &\lnot\lnot h \rightarrow \lnot l\tag{given}\\
    &h \rightarrow \lnot l \tag{double negation} \\
    &\lnot(h \rightarrow \lnot l) \tag{negation}\\
    &\lnot(\lnot h \lor\lnot l) \tag{conditional disjunction law} \\
    &h \land l \tag{de morgans law} \\
    &\text{You raised the sail and you walk the plank.}
\end{align*}
\end{mdframed}
\end{enumerate}
\item \pt{6} Give the converse, contrapositive, and inverse of the statement
``I get a treasure chest whenever I find the X on the map." (You can change tense
as needed.)
\begin{mdframed}
    \begin{align*}
        &\text{t = I get a treasure chest} \\
        &\text{x = I find x on the map} \\
        &x \rightarrow t \tag{given} \\
    \end{align*}
Contrapositve: $\lnot t \rightarrow \lnot x$ \\
Contrapositive: If I do not get a treasure chest, I do not find X on the map. \\
Converse: $t \rightarrow x$ \\
Converse: If I get a treasure chest, I find X on the map. \\
Inverse: $\lnot x \rightarrow \lnot t$ \\
Inverse: If I do not find X on the map, I do not get a treasure chest \\
\end{mdframed}
\item \pte{8} Construct truth tables for the following propositions. Include
all intermediate columns, in an appropriate order, for full credit.
\begin{enumerate}
\item $\lnot p \rightarrow (\lnot q \lor \lnot p)$
\begin{displaymath}
    \begin{array}{|c c |c c| c| c|}
    p & q & \lnot p & \lnot q & \lnot q \lor \lnot p & \lnot p \rightarrow (\lnot q \lor \lnot p)\\
    \hline 
    T & T & F & F & F & T\\
    T & F & F & T & T & T\\
    F & T & T & F & T & T\\
    F & F & T & T & T & T\\
    \end{array}
\end{displaymath}
\item $(\lnot p \lor q) \rightarrow r$
\item $(p \land \lnot q) \leftrightarrow \lnot (p \lor q)$
\end{enumerate}
\item \pte{8} Simplify each of the following to $p$, $q$, $\neg p$, $\neg q$,
$T$, or $F$ using logical equivalences. State the equivalence used at each step. Do
not skip steps. You can only use one equivalence or definition per step (even if
the same one can be applied multiple times). Do not forget about the double
negation law.
\begin{enumerate}
\item $ \lnot q \rightarrow (p \vee \lnot q)$
\begin{mdframed}
    \begin{align*}
        &\lnot q \rightarrow (p \vee \lnot q) \tag{given} \\
        &q \lor (p \lor \lnot q) \tag{conditional disjunction law} \\
        &p \lor (q \lor \lnot q) \tag{associative and commutative law} \\
        &p \lor T \tag{negation law} \\
        &T \tag{domination law} \\
    \end{align*}
\end{mdframed}
\item $(\lnot p \rightarrow q) \wedge (q \rightarrow p)$
\begin{mdframed}
    \begin{align*}
        &(\lnot p \rightarrow q) \wedge (q \rightarrow p) \tag{given} \\
        &(p \lor q) \land (\lnot q \lor p) \tag{conditional disjunction law} \\
        &p \lor (q \land \lnot q) \tag{some law idk} \\
        &p \lor F \\
        &p \\
    \end{align*}
\end{mdframed}
\end{enumerate}
\item \pte{10} Prove that $(\lnot p\wedge \lnot q) \rightarrow \lnot q \equiv
(\lnot p\wedge q) \rightarrow q$ in both of the following ways.
\begin{enumerate}
\item truth table (do not skip intermediate columns)
\item logical equivalences (do not skip steps or combine steps)
\end{enumerate}
\item \pt{8} Vikings always tell the truth and Saxons always lie. Given the
following information, use a truth table to determine what type each person is or
if their status cannot be determined. Be sure to provide a conclusion based on
your work.
Person A says: ``I am a Viking and B is a Saxon."\\
Person B says: ``C is a Saxon if A is a Viking."\\
Person C says: ``I am a Saxon or A is a Saxon only if B is a Viking."
\end{enumerate}
\end{justify}
\end{document}
