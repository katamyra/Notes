\documentclass[11pt]{article}
\usepackage[utf8]{inputenc}
\usepackage{setspace}
\setlength{\parindent}{0px}
\usepackage{sqrcaps}
\usepackage[T1]{fontenc}
\usepackage{cancel}
\usepackage{framed}
\usepackage{mdframed}
\usepackage{graphicx, tipa}
\usepackage[dvipsnames]{xcolor}
\usepackage{asymptote}
\usepackage{hyperref}
\usepackage{fancyhdr}
\usepackage{amsfonts}
\usepackage[document]{ragged2e}
\usepackage[total={6.5in, 8in}]{geometry}
\usepackage{amsmath, amsfonts, amssymb, mathtools, versions}
\usepackage{fourier}
\usepackage{adjustbox}
\usepackage{hyperref}
\usepackage{float}
\usepackage{anyfontsize}
\begin{asydef}
usepackage("fouriernc");
\end{asydef}
\newcommand{\overarc}[1]{{\setbox9=\hbox{#1}\ooalign{\resizebox{\wd9}{\height}{\
texttoptiebar{\phantom{A}}}\cr#1}}}
\DeclareMathOperator{\arcsec}{arcsec}
\DeclareMathOperator{\arccot}{arccot}
\DeclareMathOperator{\arccsc}{arccsc}
\DeclareMathOperator{\lcm}{lcm}
%For bolded items in enumerate
\newenvironment{enumbf}{\begin{enumerate}[font=\textbf]}{\end{enumerate}}
%For answer choices
\newcommand{\ans}[5]{\bigskip
$\textbf{(A)}\ #1 \qquad\textbf{(B)}\ #2 \qquad\textbf{(C)}\ #3 \qquad\textbf{(D)}\
#4 \qquad\textbf{(E)}\ #5$}
%For colored boxed answer
\newcommand{\bboxed}[1]{\color{maincolor}\boxed{#1}}
%For solutions
\newenvironment{solution}
{
\vspace{0.5mm}
\begin{mdframed}
[linewidth=3pt,leftline=true,rightline=false,bottomline=false,topline=false,linecol
or=maincolor,backgroundcolor=maincolor!10]\color{maincolor}\textbf{\
textsf{Solution: }}\color{black}
}
{\color{black}\end{mdframed}\vspace{0.5mm}}
%For alternate solutions
\newcommand{\orr}{
\begin{center}
\textbf{OR}
\end{center}
}
%For remarks
\newcommand{\bbold}[1]{\color{maincolor}\textbf{\textsf{#1} }\color{black}}
%For problem proposers
\newcommand{\prop}[1]{\color{maincolor}\textbf{\textsf{(#1)} }\color{black} }
%For headings (manual bigskips)
\newcommand{\bhead}[1]{\textbf{\textsf{\Large #1}}\normalsize}
%For headings (automatic bigskips)
\newcommand{\bbhead}[1]{\bigskip\textbf{\textsf{\Large #1}}\normalsize\bigskip}
\newcommand{\pt}[1]{\textcolor{maincolor}{(#1 points)}}
\newcommand{\pte}[1]{\textcolor{maincolor}{(#1 points each)}}
\title{CS 2050 Fall 2023 Homework 5}
\author{Due: October 6 @ 11:59 PM}
\date{Released: September 29}
\definecolor{maincolor}{RGB}{114, 140, 255}
\begin{document}
\maketitle
\begin{justify}
This assignment is due at \textbf{11:59 PM EDT} on \textbf{Friday, October 6,
2023}. Submissions submitted at least 24 hours prior to the due date will receive
2.5 points of extra credit. On-time submissions receive no penalty. You may turn it
in one day late for a 10-point penalty or two days late for a 25-point penalty.
Assignments more than two days late will NOT be accepted. We will prioritize on-
time submissions when grading before an exam.
\bigskip
You should submit a typeset or \emph{neatly} written PDF on Gradescope. The
grading TA should not have to struggle to read what you've written; if your
handwriting is hard to decipher, you will be required to typeset your future
assignments. Illegible solutions will be given 0 credit. A 5-point penalty will
occur if pages are incorrectly assigned to questions when submitting.
\bigskip
You may collaborate with other students, but any written work should be your own.
Write the names of the students you work with on the top of your assignment.
\bigskip
Always justify your work, even if the problem doesn't specify it. It can help the
TA's to give you partial credit.
\bigskip
Author(s): Kavya Selvakumar, Anthony Zang, Taiki Aiba, Atharva Gorantiwar
\clearpage
\begin{enumerate}
\item \pte{2} Determine whether the following statements are true or false. You
do not need to justify your answer.
\begin{enumerate}
\item $\emptyset \subset \emptyset$
\begin{mdframed}
    \textbf{Solution:} False.
\end{mdframed}
\item $a \in \{a\} $
\begin{mdframed}
    \textbf{Solution:} True. 
\end{mdframed}
\item $\emptyset \subseteq \{a\}$
\begin{mdframed}
    \textbf{Solution:} True.
\end{mdframed}
\item $\emptyset \in \{a\}$
\begin{mdframed}
    \textbf{Solution:} False.
\end{mdframed}
\item $\{\emptyset\} \in \{\emptyset\}$
\begin{mdframed}
    \textbf{Solution:} False.
\end{mdframed}
\item $\{\{\emptyset\}\} \subseteq \{\{\emptyset\},\{\emptyset\}\}$
\begin{mdframed}
    \textbf{Solution:} True.
\end{mdframed}
\end{enumerate}
\item \pte{3} Determine the cardinality of the following sets. You do not need
to justify your answer.
\begin{enumerate}
\item $\emptyset$
\begin{mdframed}
    \textbf{Solution:} 0.
\end{mdframed}
\item $\{T\}$
\begin{mdframed}
    \textbf{Solution:} 1.
\end{mdframed}
\item $\{\emptyset, \{\emptyset\}\}$
\begin{mdframed}
    \textbf{Solution:} 2.
\end{mdframed}
\item $\{a, \{a\},\{a, \{a\}\}\}$
\begin{mdframed}
    \textbf{Solution:} 3.
\end{mdframed}
\end{enumerate}
\item \pte{3} Find the power set of each of the following sets. You do not need
to justify your answer.
\begin{enumerate}
\item $\{a,b\}$
\begin{mdframed}
    $\{\emptyset, \{a\}, \{b\}, \{a, b\}\}$
\end{mdframed}
\item $\{1, \{2\}, 3\}$
\begin{mdframed}
    $\{\emptyset, \{1\}, \{\{2\}\}, \{3\}, \{1, \{2\}\}, \{1, 3\}, \{\{2\}, 3\}, \{1, \{2\}, 3\}\}$
\end{mdframed}
\item $\{\emptyset, \{\emptyset, \{\emptyset\}\}\}$
\begin{mdframed}
    $\{\emptyset, \{\emptyset, \{\emptyset\}\}, \{\{\emptyset, \{\emptyset\}\}\}, \{\emptyset, \{\emptyset, \{\emptyset\}\}\}\}$
\end{mdframed}
\end{enumerate}     
\item \pte{2} Find the truth set of each of the following predicates, where the
domain is the set of integers. You do not need to justify your answer.
\begin{enumerate}
\item $Q(x): x^2 < 16$
\begin{mdframed}
    $\{-3, -2, -1, 0, 1, 2, 3\}$
\end{mdframed}
\item $S(x): 3x - 5 = 9$
\begin{mdframed}
    $\{\emptyset\}$
\end{mdframed}
\end{enumerate}
\item \pte{3} Let $A = \{1, 2, 3, 4\}$ and $B = \{y,z\}$. Find the following
sets. You do not need to justify your answer.
\begin{enumerate}
\item $A \times B$
\begin{mdframed}
    $\{\{1, y\}, \{1, z\}, \{2, y\}, \{2, z\}, \{3, y\}, \{3, z\}, \{4, y\}, \{4, z\}\}$
\end{mdframed}
\item $B \times A$
\begin{mdframed}
    $\{\{y, 1\}, \{y, 2\}, \{y, 3\}, \{y, 4\}, \{z, 1\}, \{z, 2\}, \{z, 3\}, \{z, 4\}\}$
\end{mdframed}
\end{enumerate}
\newpage
\item \pte{3} Let $A = \{a,b,c,d\} $ and $B = \{a,b,c,d,e,f,g,h,i\}$, and let
$U$ represent a universal set $\{a,b,c,d,e,f,g,h,i,j,k,l\}$. Find the following
sets. You do not need to justify your answer.
\begin{enumerate}
\item $A \cup B \cup \emptyset$
\begin{mdframed}
    $\{a, b, c, d, e, f, g, h, i\}$
\end{mdframed}
\item $A - B$
\begin{mdframed}
    $\{\emptyset\} $
\end{mdframed}
\item $B - A$
\begin{mdframed}
    $\{e, f, g, h, i\}$
\end{mdframed}
\item $A^C$
\begin{mdframed}
    $\{e, f, g, h, i, j, k, l\}$
\end{mdframed}
\end{enumerate}
\item \pte{2} Suppose that $A$ is the set of Swifties at Georgia Tech and $B$
is the set of students who are majoring in computer science (CS). Express each of
these sets in terms of $A$ and $B$. This should be done with set operations, not
set-builder notation. You do not need to justify your answer.
\begin{enumerate}
\item The set of Swifties majoring in CS at Georgia Tech.
\begin{mdframed}
    $A \cap B$
\end{mdframed}
\item The set of Swifties at Georgia Tech who are not majoring in CS.
\begin{mdframed}
    $A - B$
\end{mdframed}
\item The set of students at Georgia Tech who are not Swifties and not
majoring in CS.
\begin{mdframed}
    $\overline{A} \cap \overline{B}$
\end{mdframed}
\item The set of student at Georgia Tech who are either Swifties or
majoring in CS.
\begin{mdframed}
    $A \cup B$ 
\end{mdframed}
\end{enumerate}
\item \pte{4} Prove or disprove the following statements, for all sets $A$,
$B$, and $C$ such that $A$, $B$, and $C$ are pairwise disjoint sets. You may use
set-builder notation and laws of logical equivalence; you are not permitted to use
set identities.
\begin{enumerate}
\item $A \cup (B \cup A) = A$
\begin{mdframed}
    I proceed with a direct proof to prove $A \cup (B \cup A) = A$.
    \begin{align*}
        &Statement \tag{Reason} \\
        1) &A \cup (B \cup A) = \{\, x \mid x \in A \cup (B \cup A) \,\} \tag{1. given} \\
        2) &A \cup (B \cup A) = \{\, x \mid x \in A \lor x \in (B \cup A) \,\} \tag{2. defintion of union} \\
        3) &A \cup (B \cup A) = \{\, x \mid x \in A \lor (x \in B \lor x \in A) \,\} \tag{3. definition of union} \\
        4) &A \cup (B \cup A) = \{\, x \mid (x \in A \lor x \in A) \lor x \in B\,\} \tag{4. commutative and associative law} \\
        5) &A \cup (B \cup A) = \{\, x \mid x \in A \lor x \in B \,\} \tag{5. idempotent law} \\
        6) &A \cup (B \cup A) = \{\, x \mid x \in (A \cup B)\} \tag{6. definition of set} \\
        7) &A \cup (B \cup A) = A \cup B \tag{7. definition of set and set builder} \\
    \end{align*}
    We have proven that $A \cup (B \cup A) = A$ is equal to $A \cup B$. Since it is given that the two sets \\ are disjoint
    this means that $A \cup B$ is not equal to $A$. Thus, we have direcly proven that $A \cup (B \cup A)$ is not equal to
    $A$. \\
\end{mdframed}
\item $\mathbb{P}(A) \cup \mathbb{P}(B)= \mathbb{P}(A \cup B)$
\begin{mdframed}
    I proceed with a direct proof showing that both sides are subsets of each other. Let A and B be sets.
    \begin{align*}
        &Statement \tag{Reason} \\
        1) &P(A \cup B) \tag{1. Given} \\
        2) &Let x \in P(A \cup B) \tag{2. def of arbitrary element} \\
        3) &x \subseteq A \cup B \tag{3. def of power set} \\
        4) &x \subseteq A \lor x \subseteq B \tag{4. def of union} \\
        5) &x \in P(A) \lor  x \in P(B) \tag{5. def of power set} \\
        6) &x \in P(A) \cup P(B) \tag{6. def of union} \\
        7) &P(A \cup B) \subseteq P(A) \cup P(B) \tag{7. def of subset} \\
    \end{align*}
    This, by assuming $x \in P(A \cup B)$, we have proved directly that $P(A \cup B) \subseteq P(A) \cup P(B)$.
    \begin{align*}
        &Statement \tag{Reason} \\
        1) &P(A) \cup P(B) \tag{1. Given} \\
        2) &Let x \in P(A) \cup P(B) \tag{2. def of arbitrary element} \\
        3) &x \in P(A) \lor x \in P(B) \tag{3. def of union} \\
        4) &x \subset A \lor x \in P(B) \tag{4. def of power set} \\
        5) &x \subseteq A \lor x \subseteq B \tag{5. def of power set} \\
        6) &x \subseteq (A \cup B) \tag{6. definition of union} \\
        7) &x \subseteq P(A \cup B) \tag{7. def of power set} \\
        8) &P(A) \cup P(B) \subseteq P(A \cup B) \tag{8. def of subset} \\
    \end{align*}
    This, by assuming $x \in P(A) \cup P(B)$, we have proved directly that $P(A) \cup P(B) \subseteq P(A \cup B)$. \\
    Because both the sets $P(A) \cup P(B)$ and $P(A \cup B)$ are subsets of each other, we have directly proven that they are equal.
\end{mdframed}
\item $\overline{A \cup B} = \overline{A} \cap \overline{B}$
\begin{mdframed}
    We proceed with a direct proof to prove $\overline{A \cup B} = \overline{A} \cap \overline{B}$ assuming A and B are sets.
    \begin{align*}
        &Statement \tag{Reason} \\
        1) &\overline{A \cup B} = \{\, x \mid x \in \overline{A \cup B} \} \tag{1. Given} \\
        2) &\overline{A \cup B} = \{\, x \mid x \notin A \cup B \} \tag{2. def of complement} \\
        3) &\overline{A \cup B} = \{\, x \mid \neg(x \in A \cup B)\} \tag{3. def of not element in } \\
        4) &\overline{A \cup B} = \{\, x \mid \neg(x \in A \lor x \in B)\} \tag{4. def of union} \\
        5) &\overline{A \cup B} = \{\, x \mid \neg(x \in A) \land \neg(x \in B)\} \tag{5. De Morgan's Law} \\
        6) &\overline{A \cup B} = \{\, x \mid x \notin A \land \neg(x \in B)\} \tag{6. def of not element in} \\
        7) &\overline{A \cup B} = \{\, x \mid x \notin A \land x \notin B\} \tag{7. def of not element in} \\
        8) &\overline{A \cup B} = \{\, x \mid x \in \overline{A} \land x \notin B\} \tag{8. def of complement} \\
        9) &\overline{A \cup B} = \{\, x \mid x \in \overline{A} \land x \in \overline{B}\} \tag{9. def of complement} \\
        10) &\overline{A \cup B} = \overline{A} \cap \overline{B} \tag{10. def of intersection} \\
    \end{align*}
    This, we have directly proven that $\overline{A \cup B} = \overline{A} \cap \overline{B}$.
\end{mdframed}
\end{enumerate}
\item \pt{5} List all the elements of $\mathbb{P}(\mathbb{P}(\mathbb{P}(\emptyset)))-\mathbb{P}(\mathbb{P}(\emptyset))\cup \mathbb{P}(\emptyset)-\emptyset$.
You do not need to justify your answer.
\begin{mdframed}
    $P(\emptyset) = \{\emptyset\}$ \\
    $P(P(\emptyset)) = \{ \emptyset, \{ \emptyset\}\}$ \\
    $P(P(\emptyset)) \cup P(\emptyset) = \{\emptyset, \{\emptyset\}\}$ \\
    $P(P(P(\emptyset))) = \{ \emptyset, \{ \emptyset\}, \{ \{ \emptyset\}\}, \{ \emptyset, \{ \emptyset\}\}\}$ \\
    $P(P(P(\emptyset))) - P(P(\emptyset)) \cup P(\emptyset) - \emptyset = 
    \{\emptyset,\{\{ \emptyset\}\}, \{\emptyset, \{\emptyset\}\} \}$ \\
\end{mdframed}
\item \pt{8} Prove that: \{$5a + 6b$ | $a,b \in \mathbb{Z}$\} $ = \mathbb{Z}$
\begin{mdframed}
    In order to prove that \{$5a + 6b$ | $a,b \in \mathbb{Z}$\} $ = \mathbb{Z}$, we will prove that each are subsets of each other.
    \begin{align*}
        &Statement \tag{Reason} \\
        1) &\{5a + 6b \,|\, a,b \in \mathbb{Z}\} \tag{1. Given}\\
        2) &j = 5a+6b \text{\;for some j $\in \mathbb{Z}$} \tag{2. closure of add and mult in Z}\\
        3) &\text{Since j $\in \mathbb{Z}$, }\;\{5a + 6b \,| \,a,b \in \mathbb{Z}\} \subseteq \mathbb{Z} \tag{3. def of subset and combine (1) and (2)} \\
        &Statement \tag{Reason} \\
        1) &j=5a+6b\, \text{for some j $\in \mathbb{Z}$} \tag{1. Given} \\
        2) &\text{Because $j \in \mathbb{Z}$, $\mathbb{Z} \subseteq \{5a + 6b \,|\, a,b \in \mathbb{Z}\}$} \tag{2. def of subset with (1)} \\
    \end{align*}
    Because both sets are subsets of each other, they are equivalent and thus we have directly proven that \{$5a + 6b$ | $a,b \in \mathbb{Z}$\} $ = \mathbb{Z}$.
\end{mdframed}
\item \pte{2} For each of the following maps, determine whether $f$ is onto,
one-to-one, and/or a function. $A = \{a, b, c\}$, $B = \{a, b, c, d\}$, $C = \{1,
2, 3\}$, and $D = \{1, 2, 3, 4\}$. You do not need to justify your answer.
\begin{enumerate}
\item $f: A \to C$, $f(a) = 2, f(b) = 3, f(c) = 1$
\begin{mdframed}
    \textbf{Solution:} Function, one to one, and onto.
\end{mdframed}
\item $f: A \to D$, $f(a) = 1, f(b) = 2, f(c) = 3$
\begin{mdframed}
    \textbf{Solution:} One-to-one and a function.
\end{mdframed}
\item $f: B \to C$, $f(a) = 3, f(b) = 2, f(c) = 2, f(d) = 1$
\begin{mdframed}
    \textbf{Solution:} Function, onto.
\end{mdframed}
\item $f: B \to D$, $f(a) = 1, f(a) = 2, f(b) = 3, f(b) = 4$
\begin{mdframed}
    \textbf{Solution:} Not a function, onto.
\end{mdframed}
\end{enumerate}
\item \pte{2} For each of the following, find the inverse function $f^{-1}$ or
state why it does not exist. You do not need to justify your answer.
\begin{enumerate}
\item $f: \mathbb{R} \to \mathbb{R}$, $f(x) = \frac{1}{x}$
\begin{mdframed}
    \textbf{Solution:} The inverse doesn't exist because $f(0)$ is undetermined.
\end{mdframed}
\item $f: \mathbb{R} \to \mathbb{R}$, $f(x) = 5x - 6$
\begin{mdframed}
    $f^-1(x) = \frac{x+6}{5}$
\end{mdframed}
\item $f: \mathbb{R} \to \mathbb{R}$, $f(x) = |x|$
\begin{mdframed}
    \textbf{Solution: }Inverse is not a function as a absolute value will cause the same input value to have multiple outpue values.
\end{mdframed}
\end{enumerate}
\end{enumerate}
\end{justify}
\end{document}
