\documentclass[11pt]{article}
\usepackage[utf8]{inputenc}
\usepackage{setspace}
\setlength{\parindent}{0px} 
\usepackage{sqrcaps}
\usepackage[T1]{fontenc}
\usepackage{cancel}  
\usepackage{framed} 
\usepackage{mdframed}
\usepackage{graphicx, tipa}
\usepackage[dvipsnames]{xcolor}
\usepackage{asymptote}
\usepackage{hyperref}
\usepackage{fancyhdr}
\usepackage{amsfonts}
\usepackage[document]{ragged2e}
\usepackage[total={6.5in, 8in}]{geometry}
\usepackage{amsmath, amsfonts, amssymb, mathtools, versions}
\usepackage{mdframed}
\usepackage{fourier}
\usepackage{adjustbox}
\usepackage{hyperref}
\usepackage{float}
\usepackage{anyfontsize}
\begin{asydef}
usepackage("fouriernc");https://www.overleaf.com/project/64e4e4d60e66a062b4a2e9d1
\end{asydef}
\newcommand{\overarc}[1]{{\setbox9=\hbox{#1}\ooalign{\resizebox{\wd9}{\height}{\texttoptiebar{\phantom{A}}}\cr#1}}}
\DeclareMathOperator{\arcsec}{arcsec}
\DeclareMathOperator{\arccot}{arccot}
\DeclareMathOperator{\arccsc}{arccsc}
\DeclareMathOperator{\lcm}{lcm}

%For bolded items in enumerate
\newenvironment{enumbf}{\begin{enumerate}[font=\textbf]}{\end{enumerate}}

%For answer choices
\newcommand{\ans}[5]{\bigskip

$\textbf{(A)}\ #1 \qquad\textbf{(B)}\ #2 \qquad\textbf{(C)}\ #3 \qquad\textbf{(D)}\ #4 \qquad\textbf{(E)}\ #5$}

%For colored boxed answer
\newcommand{\bboxed}[1]{\color{maincolor}\boxed{#1}}

%For solutions
\newenvironment{solution}
{
\vspace{0.5mm}
\begin{mdframed}[linewidth=3pt,leftline=true,rightline=false,bottomline=false,topline=false,linecolor=maincolor,backgroundcolor=maincolor!10]\color{maincolor}\textbf{\textsf{Solution: }}\color{black}

}
{\color{black}\end{mdframed}\vspace{0.5mm}}

%For alternate solutions
\newcommand{\orr}{
\begin{center}
\textbf{OR}
\end{center}
}

%For remarks
\newcommand{\bbold}[1]{\color{maincolor}\textbf{\textsf{#1} }\color{black}} 

%For problem proposers
\newcommand{\prop}[1]{\color{maincolor}\textbf{\textsf{(#1)} }\color{black} } 

%For headings (manual bigskips)
\newcommand{\bhead}[1]{\textbf{\textsf{\Large #1}}\normalsize}

%For headings (automatic bigskips)
\newcommand{\bbhead}[1]{\bigskip\textbf{\textsf{\Large #1}}\normalsize\bigskip} 

\newcommand{\pt}[1]{\textcolor{maincolor}{(#1 points)}}

\newcommand{\pte}[1]{\textcolor{maincolor}{(#1 points each)}}

\title{CS 2050 Fall 2023 Homework 2}
\author{Due: September 8 @ 11:59 PM}
\date{Released: September 1}

\definecolor{maincolor}{RGB}{114, 140, 255} 

\begin{document}

\maketitle


\begin{justify}



This assignment is due at \textbf{11:59 PM EDT} on \textbf{Friday, September 8, 2023}. Submissions submitted at least 24 hours prior to the due date will receive 2.5 points of extra credit. On-time submissions receive no penalty. You may turn it in one day late for a 10-point penalty or two days late for a 25-point penalty. Assignments more than two days late will NOT be accepted.  We will prioritize on-time submissions when grading before an exam.

\bigskip

You should submit a typeset or \emph{neatly} written PDF on Gradescope.  The grading TA should not have to struggle to read what you've written; if your handwriting is hard to decipher, you will be required to typeset your future assignments. Illegible solutions will be given 0 credit. A 5-point penalty will occur if pages are incorrectly assigned to questions when submitting.

\bigskip

You may collaborate with other students, but any written work should be your own. Write the names of the students you work with on the top of your assignment.

\bigskip

Always justify your work, even if the problem doesn't specify it. It can help the TA's to give you partial credit.

\bigskip

Author(s): Anthony Zang, Taiki Aiba, Atharva Gorantiwar, Kavya Selvakumar

\clearpage

\begin{enumerate}

    \item \pte{3} Express each of the following statements using predicates and quantifiers.

    \begin{enumerate}
        \item ``Not all bakers bake baguettes.''\\
              Let the domain be the set containing all people.\\
              Let $B(x):$ $x$ is a baker.\\
              Let $A(x):$ $x$ bakes a baguette.
        \begin{mdframed}
            $\lnot \forall x (B(x) \rightarrow A(x))$ 
        \end{mdframed}

        \item ``Some movies are scary.''\\
               Let the domain be the set containing all videos.\\
               Let $M(x):$ $x$ is a movie. \\
               Let $S(x):$ $x$ is scary.
        \begin{mdframed}
            $\exists x (M(x) \land S(x))$ 
        \end{mdframed}
        
        \item ``All parallelograms are polygons but not all polygons are parallelograms.''\\
                Let the domain be the set of all shapes.\\
                Let P(x): x is a polygon.\\
                Let T(x): x is a parallelogram.
        \begin{mdframed}
            $\forall x (T(x) \rightarrow P(x)) \land \lnot \forall x (P(x) \rightarrow T(x))$
        \end{mdframed}

        \item ``All Taylor Swift songs are \#1 hits.''\\
               Let the domain be the set containing all music.\\
               Let $T(x):$ $x$ is a Taylor Swift song.\\
               Let $H(x):$ $x$ is a \#1 hit.
        \begin{mdframed}
            $\forall x (T(x) \rightarrow H(x))$
        \end{mdframed}
        
    \end{enumerate}
    
    \item \pte{5} For each of the following statements, push all negations $(\neg)$ as far as possible so that no negation is to the left of a quantifier (in other words, the negation is immediately to the left of a predicate). Show your work.

    \begin{enumerate}
        \item $\neg \exists x \neg \forall y \neg \forall z (A(x,z) \land B(y,z))$
        \begin{mdframed}
            $\neg \exists x \neg \forall y \exists z \neg (A(x, z) \land B(y, z))$ \\
            $\neg \exists x \exists y \forall z \neg \neg (A(x, z) \land B(y, z))$ \\
            $\forall x  \forall y \exists z \neg (A(x, z) \land B(y, z))$ \\
            $\forall x  \forall y \exists z (\neg A(x, z) \lor \neg B(y, z))$
        \end{mdframed}
        \item $\neg  \forall y (\exists x A(x,y) \rightarrow \exists z B(z,y))$
        \item $\neg \forall x((\exists y A(x,y) \rightarrow \exists y B(x,y)) \rightarrow \neg \exists y C(x,y))$
        \begin{mdframed}
            $ \exists x \neg (\neg (\exists y A(x, y) \rightarrow \exists y B(x,y)) \lor \neg \exists y C(x, y))$ \\
            $ \exists x \neg (\neg (\neg \exists y A(x,y) \lor \exists y B(x,y)) \lor \neg \exists y C(x, y))$ 
        \end{mdframed}
    \end{enumerate}

    \item \pte{4} Let $K(x)$ be the statement ``$x$ has a pet koala,'' let $G(x)$ be the statement ``$x$ has a gazelle,'' let $U(x)$ be the statement ``$x$ has a unicorn,'' and let $F(x)$ be the statement ``$x$ has a farm.'' Express each of these statements in terms of $K(x)$, $G(x)$, $U(x)$, $F(x)$, quantifiers, and logical connectives. Let the domain consist of all people.

    \begin{enumerate}
        \item Everyone owns a gazelle or a unicorn.
        \begin{mdframed}
            $\forall x (G(x) \lor U(x))$
        \end{mdframed}
        \item Any person who does not own a farm cannot own a koala, gazelle, or unicorn.
        \begin{mdframed}
            $\forall x (\neg F(x) \rightarrow \neg (K(x) \lor G(x) \lor U(x)))$
        \end{mdframed}
        \item No one owns a gazelle, but everyone owns a koala.
        \begin{mdframed}
            $\neg \forall xG(x) \land \forall xK(x)$
        \end{mdframed}
        \item Some people own both a unicorn and a farm, but not all people own both a koala and a gazelle.
        \begin{mdframed}
            $\exists x(U(x) \land F(x)) \land \neg \forall x(K(x) \land G(x))$
        \end{mdframed}
        \item Each type of animal (koala, gazelle, unicorn) is owned by at least one person.
        \begin{mdframed}
            $\exists xK(x) \land \exists yG(y) \land \exists zU(z)$
        \end{mdframed}
    \end{enumerate}

    \item \pte{4} Determine the truth value of each of these statements given that the domain of each variable consists of all real numbers. If true, give a brief explanation explaining why it is true. If false, give a counterexample (i.e., give values $x$, and $y$ that make the statement false; you must also explain why they make the statement false).

    \begin{enumerate}
        \item $\forall y \exists x(x \cdot y > 0)$ 
        \begin{mdframed}
            False. If y is zero, no choice of x will make xy > 0
        \end{mdframed}
        \item $\forall x \exists y((x > y) \rightarrow (x^2 > y^2))$
        \begin{mdframed}
            True. For any value of x, make y larger so that the implication is always true since the hypothesis is false.
        \end{mdframed}
        \item $\forall y \forall x((x > y) \rightarrow (x^2 > y^2))$
        \begin{mdframed}
            False. If x is -1 and y is -2, then $x^2$ is not greater than $y^2$
        \end{mdframed}
        \item $\exists x \forall y((y > 0) \rightarrow (x^y \geq y))$
        \begin{mdframed}
            True. If y > 0, $x^y$ will always be greater or equal to y
        \end{mdframed}
        \item $\forall y \exists ! x(\sqrt[x]{y} < 0)$
        \begin{mdframed}
            False, y = 4, x = 2
        \end{mdframed}
        \item $\exists x \forall y ((x + y > 0) \rightarrow (y > 0))$
        \begin{mdframed}
            True. existential player can choose x = 0, so y has to be > 0 for True True, or if y < 0, left\\  side of
            implication is false making the whole thing True.
        \end{mdframed}
    \end{enumerate}

    \item \pt{10} Show that $\neg \forall y \exists x ((P(y) \land \neg Q(x)) \lor R(x, y))$ is logically equivalent to $\exists y \forall x ((P(y) \rightarrow Q(x)) \land \neg R(x, y))$. Make sure to cite all steps (e.g. De Morgan's Law). You should neither combine steps nor skip steps, even if two steps are of the same law. The only steps that can be combined on a single line are the associative and commutative laws.
    \begin{mdframed}
        \begin{align*}
            &\neg \forall y \exists x ((P(y) \land \neg Q(x))  \lor R(x, y)) \tag{given} \\
            &\exists y \forall x \neg ((P(y) \land \neg Q(x))  \lor R(x, y)) \tag{de morgans law for quantifiers} \\
            &\exists y \forall x (\neg (P(y) \land \neg Q(x)) \land \neg R(x, y)) \tag{de morgans law} \\
            &\exists y \forall x ((\neg P(y) \lor Q(x)) \land \neg R(x,y)) \tag{de morgans law} \\
            &\exists y \forall x ((P(y) \rightarrow Q(x)) \land \neg R(x, y)) \tag{conditional disjunction law} \\
        \end{align*}
        We have now proven that both sides are logically equivelent as we were able to reach one statement through the other
    \end{mdframed}

    \item \pte{4} For each of these universally quantified statements, find a counterexample. If there are none, then state that no counterexample exists. The domain for all variables is all real numbers.
    
    \begin{enumerate}
        \item $\forall x \exists y (y = \frac{x^3}{5})$
        \begin{mdframed}
            No counterexample.
        \end{mdframed}
        \item $\forall x ((x+1)^2 = x^2 + 1)$
        \begin{mdframed}
            x = 2 is a counter-example.
        \end{mdframed}
        \item $\forall x \forall y ((x \geq y) \rightarrow ((x^{50} > y) \lor (y \geq 0)))$
        \begin{mdframed}
            If x is greater y, it has to be true that $x^50$ is greater than y or y is greater than or equal to zero.
        \end{mdframed}
    \end{enumerate}

    \item \pt{7} Use predicates, quantifiers, logical connectives, and mathematical operators to express the statement that: "there is a positive integer that is not the sum of three squares."
    \begin{mdframed}
        Domain: integer \\
        $\exists a \exists b \exists c \exists d (a > 0 \land \lnot (a = b^2+c^2+d^2))$
    \end{mdframed}

    
    
\end{enumerate}
\end{justify}
\end{document}